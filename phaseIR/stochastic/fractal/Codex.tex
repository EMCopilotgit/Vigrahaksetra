\chapter{Fractal Time}

\begin{flushright}
\textit{“[Optional Epigraph: a poetic fragment, mythic line, or mathematical aphorism]”} \\
— [Attribution, if any]
\end{flushright}

\section*{Bridge}
This chapter extends the lineage of PhaseIR into the domain of \textbf{Fractal} time.  
Here, the operators are not merely technical primitives but glyphs of a deeper chronicity.  
The commentary that follows binds symbolic resonance with reproducible form, ensuring that  
mythos and mechanism remain inseparable.

\section{Technical Operators}
\begin{itemize}
  \item \texttt{Operator1.v} — short description of its role.
  \item \texttt{Operator2.v} — short description of its role.
  \item \texttt{OperatorN.v} — etc.
\end{itemize}

\section{Mythic Commentary}
Here the Codex voice expands:  
- Describe the symbolic resonance of this domain (e.g. elliptic time as “periodicity with hidden modulus”).  
- Tie the operators to archetypes (e.g. JacobiSn as “the sine that remembers its modulus”).  
- Use metaphor, narrative, or allegory to weave the technical into the mythic.  

\section{Calibration Notes}
(Optional) Document any reproducibility practices, approximations, or calibration rituals  
associated with this domain. This ensures the Codex remains both poetic and practical.

\section{Inheritance}
Conclude with a short reflection on how this domain’s glyphs inherit from or  
interact with other domains (e.g. fractal ↔ spectral, elliptic ↔ p-adic).  
This creates continuity across chapters when compiled into the master Codex.
